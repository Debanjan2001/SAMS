%Copyright 2014 Jean-Philippe Eisenbarth
%This program is free software: you can 
%redistribute it and/or modify it under the terms of the GNU General Public 
%License as published by the Free Software Foundation, either version 3 of the 
%License, or (at your option) any later version.
%This program is distributed in the hope that it will be useful,but WITHOUT ANY 
%WARRANTY; without even the implied warranty of MERCHANTABILITY or FITNESS FOR A 
%PARTICULAR PURPOSE. See the GNU General Public License for more details.
%You should have received a copy of the GNU General Public License along with 
%this program.  If not, see <http://www.gnu.org/licenses/>.

%Based on the code of Yiannis Lazarides
%http://tex.stackexchange.com/questions/42602/software-requirements-specification-with-latex
%http://tex.stackexchange.com/users/963/yiannis-lazarides
%Also based on the template of Karl E. Wiegers
%http://www.se.rit.edu/~emad/teaching/slides/srs_template_sep14.pdf
%http://karlwiegers.com
\documentclass{scrreprt}
\usepackage{listings}
\usepackage{underscore}
\usepackage[bookmarks=true]{hyperref}
\usepackage[utf8]{inputenc}
\usepackage[english]{babel}
\hypersetup{
    bookmarks=false,    % show bookmarks bar?
    pdftitle={Software Requirement Specification},    % title
    pdfauthor={Jean-Philippe Eisenbarth},                     % author
    pdfsubject={TeX and LaTeX},                        % subject of the document
    pdfkeywords={TeX, LaTeX, graphics, images}, % list of keywords
    colorlinks=true,       % false: boxed links; true: colored links
    linkcolor=blue,       % color of internal links
    citecolor=black,       % color of links to bibliography
    filecolor=black,        % color of file links
    urlcolor=purple,        % color of external links
    linktoc=page            % only page is linked
}%
\def\myversion{1.0.0 }
\date{}
%\title
\usepackage{hyperref}
\begin{document}
\thispagestyle{empty}
\begin{flushright}
    \rule{16cm}{5pt}\vskip1cm
    \bfseries{
        \Huge{SOFTWARE REQUIREMENTS\\ SPECIFICATION}\\
        \vspace{1.9cm}
        for\\
        \vspace{1.9cm}
        Student Auditorium Management System (SAMS)\\
        \vspace{1.9cm}
        Prepared by \\
        Aaditya Agrawal (19CS10003)\\
        Debanjan Saha (19CS30014) \\
        Deep Majumder (19CS30015) \\ 
        \vspace{1.9cm}
        \today\\
}
\end{flushright}

\tableofcontents


\chapter*{Revision History}

\begin{center}
    \begin{tabular}{|c|c|c|c|}
        \hline
	    Name & Date & Reason For Changes & Version\\
        \hline
	     &  &  & \\
        \hline
    \end{tabular}
\end{center}

\chapter{Introduction}

\section{Purpose}
This document is to define the functional and non-functional requirements of the Student Auditorium Management System (SAMS). This software consists of a backend API server and a frontend application, which together allow for efficiently managing shows, selling and refunding tickets and maintaining a balance sheet.

\section{Document Conventions}

Following defintions and abbreviations are used in this document:
\begin{enumerate}
	\item SAMS: Students' Auditorium Management System
	\item SM: Show Manager
	\item SP: Sales Person
	\item AC: Accounts Clerk
	\item User: User corresponds to either SM or AC or SP or customer.
\end{enumerate}

\section{Intended Audience and Reading Suggestions}

This document mainly focuses on the design analysis and requirement specifications for this software. Software developers, testers, documentation writers are always welcome to read this document. However, it is not necessary for the users of this software to read this document. Nevertheless, we would encourage everyone to look into this document to get to know how we create the software from ground zero.
\\ \\
For the readers, we would encourage you to read the content in the same order in which we have documented this and not to skip in between, so that you get a smoother understanding of this software. 

\section{Project Scope}
$<$Provide a short description of the software being specified and its purpose, 	
including relevant benefits, objectives, and goals. Relate the software to 
corporate goals or business strategies. If a separate vision and scope document 
is available, refer to it rather than duplicating its contents here.$>$

\section{References}
$<$List any other documents or Web addresses to which this SRS refers. These may 
include user interface style guides, contracts, standards, system requirements 
specifications, use case documents, or a vision and scope document. Provide 
enough information so that the reader could access a copy of each reference, 
including title, author, version number, date, and source or location.$>$


\chapter{Overall Description}

\section{Product Perspective}
$<$Describe the context and origin of the product being specified in this SRS.  
For example, state whether this product is a follow-on member of a product 
family, a replacement for certain existing systems, or a new, self-contained 
product. If the SRS defines a component of a larger system, relate the 
requirements of the larger system to the functionality of this software and 
identify interfaces between the two. A simple diagram that shows the major 
components of the overall system, subsystem interconnections, and external 
interfaces can be helpful.$>$

\section{Product Functions}

The software is intended for the use of four types of users - show manager,accounts clerk,sales persons and customers(spectators). The functions for each type of user provided by this software are listed as follows:


\begin{enumerate}
\item \textbf{Show Manager}
	\begin{enumerate}
		\item Add New Show: The SM can add a new show with a date,timings based on the availability of the auditorium. Fixing a particular number of shows on a date and fixing the number and price of the balcony and ordinary seats for sale in a particular show are exclusive rights of SM. Offering seats as complimentary gifts to the students' society and VIPs are also handled by the SM.

		\item Manage Account for SP and AC: The SM is able to create new account for authorised SP and AC and will be able to change passwords for the all SP and AC.
		
		\item Query: The SM can query the percentage of seats booked of each type and also the amount collected for each show
		
		\item Check Balance Sheet: The SM can access different types of balance sheets(based on a show or annual sheet) for the auditorium.
		
	\end{enumerate}
	
\item \textbf{Accounts Clerk}
	\begin{enumerate}
	
		\item Add Expenditure: The AC can add expenditures for a show with a description for the expense. 
		
	\end{enumerate}

\item \textbf{Sales Person}
	\begin{enumerate}
		\item  Book Ticket: SP will be able to book and authorise tickets for a customer and make a sale for the show.
	\end{enumerate}

\item \textbf{Customer}
	\begin{enumerate}
		
		\item Cancel Tickets: The Customer will be able to cancel their tickets before the starting of the show.
		
		\item Query for Seats: The Customer will be able to check the availability of different classes of seats.
		
	\end{enumerate}
 
 
\end{enumerate}


\section{User Classes and Characteristics}
$<$Identify the various user classes that you anticipate will use this product.  
User classes may be differentiated based on frequency of use, subset of product 
functions used, technical expertise, security or privilege levels, educational 
level, or experience. Describe the pertinent characteristics of each user class.  
Certain requirements may pertain only to certain user classes. Distinguish the 
most important user classes for this product from those who are less important 
to satisfy.$>$

\section{Operating Environment}

We are testing this to work on the following tech stack- JDK11 and above, MongoDB 4.4 and above, Apache Maven 3.6 and above and node 14.5 and above. We have tested it to run on reasonably recent modern browser like Google Chrome 88 and above, firefox 86 and above with amd64 architecture.

\section{Design and Implementation Constraints}

\begin{enumerate}
	\item Total number of seats in the auditorium must be fixed at any time.
	
	\item  Total number of seats booked by a customer can not exceed a beyond a considerable logical value.
	
	\item SM can not be fired. In other words, there must be a SM present in the auditorium to manage the system.
	
	\item Tickets can not be sold by spectators or transfered to others.
	
\end{enumerate}


\section{User Documentation}

The frontend of the software will be kept sufficiently intuitive and straight forward for an internet literate person to use.

\section{Assumptions and Dependencies}

We have assumed the following: 
\begin{itemize}
	\item Computers are available to all types of users and users are computer literate and internet literate.
	
	\item Only one SM is present at a time. 
	
\end{itemize}

Our dependencies are as follows:
\begin{itemize}
	\item The software is being developed in linux operating system.
\end{itemize}


\chapter{External Interface Requirements}

\section{User Interfaces}
$<$Describe the logical characteristics of each interface between the software 
product and the users. This may include sample screen images, any GUI standards 
or product family style guides that are to be followed, screen layout 
constraints, standard buttons and functions (e.g., help) that will appear on 
every screen, keyboard shortcuts, error message display standards, and so on.  
Define the software components for which a user interface is needed. Details of 
the user interface design should be documented in a separate user interface 
specification.$>$

\section{Hardware Interfaces}
$<$Describe the logical and physical characteristics of each interface between 
the software product and the hardware components of the system. This may include 
the supported device types, the nature of the data and control interactions 
between the software and the hardware, and communication protocols to be 
used.$>$

\section{Software Interfaces}
$<$Describe the connections between this product and other specific software 
components (name and version), including databases, operating systems, tools, 
libraries, and integrated commercial components. Identify the data items or 
messages coming into the system and going out and describe the purpose of each.  
Describe the services needed and the nature of communications. Refer to 
documents that describe detailed application programming interface protocols.  
Identify data that will be shared across software components. If the data 
sharing mechanism must be implemented in a specific way (for example, use of a 
global data area in a multitasking operating system), specify this as an 
implementation constraint.$>$

\section{Communications Interfaces}
$<$Describe the requirements associated with any communications functions 
required by this product, including e-mail, web browser, network server 
communications protocols, electronic forms, and so on. Define any pertinent 
message formatting. Identify any communication standards that will be used, such 
as FTP or HTTP. Specify any communication security or encryption issues, data 
transfer rates, and synchronization mechanisms.$>$


\chapter{System Features}
$<$This template illustrates organizing the functional requirements for the 
product by system features, the major services provided by the product. You may 
prefer to organize this section by use case, mode of operation, user class, 
object class, functional hierarchy, or combinations of these, whatever makes the 
most logical sense for your product.$>$

\section{System Feature 1}
$<$Don’t really say “System Feature 1.” State the feature name in just a few 
words.$>$

\subsection{Description and Priority}
$<$Provide a short description of the feature and indicate whether it is of 
High, Medium, or Low priority. You could also include specific priority 
component ratings, such as benefit, penalty, cost, and risk (each rated on a 
relative scale from a low of 1 to a high of 9).$>$

\subsection{Stimulus/Response Sequences}
$<$List the sequences of user actions and system responses that stimulate the 
behavior defined for this feature. These will correspond to the dialog elements 
associated with use cases.$>$

\subsection{Functional Requirements}
$<$Itemize the detailed functional requirements associated with this feature.  
These are the software capabilities that must be present in order for the user 
to carry out the services provided by the feature, or to execute the use case.  
Include how the product should respond to anticipated error conditions or 
invalid inputs. Requirements should be concise, complete, unambiguous, 
verifiable, and necessary. Use “TBD” as a placeholder to indicate when necessary 
information is not yet available.$>$

$<$Each requirement should be uniquely identified with a sequence number or a 
meaningful tag of some kind.$>$

REQ-1:	REQ-2:

\section{System Feature 2 (and so on)}


\chapter{Other Nonfunctional Requirements}

\section{Performance Requirements}
$<$If there are performance requirements for the product under various 
circumstances, state them here and explain their rationale, to help the 
developers understand the intent and make suitable design choices. Specify the 
timing relationships for real time systems. Make such requirements as specific 
as possible. You may need to state performance requirements for individual 
functional requirements or features.$>$

\section{Safety Requirements}
$<$Specify those requirements that are concerned with possible loss, damage, or 
harm that could result from the use of the product. Define any safeguards or 
actions that must be taken, as well as actions that must be prevented. Refer to 
any external policies or regulations that state safety issues that affect the 
product’s design or use. Define any safety certifications that must be 
satisfied.$>$

\section{Security Requirements}
$<$Specify any requirements regarding security or privacy issues surrounding use 
of the product or protection of the data used or created by the product. Define 
any user identity authentication requirements. Refer to any external policies or 
regulations containing security issues that affect the product. Define any 
security or privacy certifications that must be satisfied.$>$

\section{Software Quality Attributes}
$<$Specify any additional quality characteristics for the product that will be 
important to either the customers or the developers. Some to consider are: 
adaptability, availability, correctness, flexibility, interoperability, 
maintainability, portability, reliability, reusability, robustness, testability, 
and usability. Write these to be specific, quantitative, and verifiable when 
possible. At the least, clarify the relative preferences for various attributes, 
such as ease of use over ease of learning.$>$

\section{Business Rules}
$<$List any operating principles about the product, such as which individuals or 
roles can perform which functions under specific circumstances. These are not 
functional requirements in themselves, but they may imply certain functional 
requirements to enforce the rules.$>$


\chapter{Other Requirements}
$<$Define any other requirements not covered elsewhere in the SRS. This might 
include database requirements, internationalization requirements, legal 
requirements, reuse objectives for the project, and so on. Add any new sections 
that are pertinent to the project.$>$

\section{Appendix A: Glossary}
%see https://en.wikibooks.org/wiki/LaTeX/Glossary
$<$Define all the terms necessary to properly interpret the SRS, including 
acronyms and abbreviations. You may wish to build a separate glossary that spans 
multiple projects or the entire organization, and just include terms specific to 
a single project in each SRS.$>$

\section{Appendix B: Analysis Models}
$<$Optionally, include any pertinent analysis models, such as data flow 
diagrams, class diagrams, state-transition diagrams, or entity-relationship 
diagrams.$>$

\section{Appendix C: To Be Determined List}
$<$Collect a numbered list of the TBD (to be determined) references that remain 
in the SRS so they can be tracked to closure.$>$

\end{document}
